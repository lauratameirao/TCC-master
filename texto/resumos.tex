% resumo em português
\setlength{\absparsep}{18pt} % ajusta o espaçamento dos parágrafos do resumo
\begin{resumo}
 % Segundo a \citeonline[3.1-3.2]{NBR6028:2003}, o resumo deve ressaltar o
 % objetivo, o método, os resultados e as conclusões do documento. A ordem e a extensão
 % destes itens dependem do tipo de resumo (informativo ou indicativo) e do
 % tratamento que cada item recebe no documento original. O resumo deve ser
 % precedido da referência do documento, com exceção do resumo inserido no
 % próprio documento. (\ldots) As palavras-chave devem figurar logo abaixo do
 % resumo, antecedidas da expressão Palavras-chave:, separadas entre si por
 % ponto e finalizadas também por ponto.

 A primeira versão deste Trabalho de Concusão de Curso consiste em uma revisão bibliográfica a respeito do tema: Otimização estrutural de um painel reforçado utilizando os parâmetros de laminação. As próximas versões irão conter mais detalhes da metodologia adotada e os resultados obtidos durante o desenvolvimento do trabalho.


 \textbf{Palavras-chave}: engenharia aeroespacial. otimização. materiais compostos. parâmetros de laminação. painel reforçado. nastran sol 200.
\end{resumo}

% % resumo em inglês
% \begin{resumo}[Abstract]
%  \begin{otherlanguage*}{english}
%    - to do -
%
%    \vspace{\onelineskip}
%
%    \noindent
%    \textbf{Keywords}: aerospace. missile. simulation. hardware-in-the-loop.
%    model based design. model based system engineering. test driven development.
%  \end{otherlanguage*}
% \end{resumo}
%
% % resumo em francês
% \begin{resumo}[Résumé]
%  \begin{otherlanguage*}{french}
%     - à faire -
%
%    \textbf{Mots-clés}: aérospatiale. missile. simulation. hardware-in-the-loop.
%    model based design. model based system engineering. test driven development.
%  \end{otherlanguage*}
% \end{resumo}
%
% % resumo em espanhol
% \begin{resumo}[Resumen]
%  \begin{otherlanguage*}{spanish}
%    - a hacer -
%
%    \textbf{Palabras clave}: aeroespacial. misil. simulación.
%    hardware-in-the-loop. model based design. model based system engineering.
%    test driven development.
%  \end{otherlanguage*}
% \end{resumo}
