% resumo em português
\setlength{\absparsep}{18pt} % ajusta o espaçamento dos parágrafos do resumo
\begin{resumo}
 % Segundo a \citeonline[3.1-3.2]{NBR6028:2003}, o resumo deve ressaltar o
 % objetivo, o método, os resultados e as conclusões do documento. A ordem e a extensão
 % destes itens dependem do tipo de resumo (informativo ou indicativo) e do
 % tratamento que cada item recebe no documento original. O resumo deve ser
 % precedido da referência do documento, com exceção do resumo inserido no
 % próprio documento. (\ldots) As palavras-chave devem figurar logo abaixo do
 % resumo, antecedidas da expressão Palavras-chave:, separadas entre si por
 % ponto e finalizadas também por ponto.

 O intuito deste Trabalho de Conclusão de Curso é o de realizar uma revisão bibliográfica a respeito dos temas de materiais compostos e otimização, bem como descrever como foi feita uma otimização estrutural de um painel reforçado utilizando os parâmetros de laminação. Sabe-se que a utilização dos materiais compostos em estruturas aeronáuticas tem aumentado gradualmente nas últimas décadas, e também, que estruturas primárias como asas e empenagens possuem a tendência de serem projetadas utilizando painéis reforçados constituídos de material composto. Isto se deve ao fato de as estruturas em materiais compostos possuírem elevados valores de resistência e rigidez específicas. As propriedades do material composto podem ser otimizadas utilizando como ferramenta os parâmetros de laminação, apresentando um potencial de uso bastante elevado para esse tipo de material. Portanto, foram desenvolvidos modelos de otimização do painel reforçado utilizando o \emph{software Nastran} para comparar os resultados entre painéis reforçados fabricados de material metálico e de material composto. O desenvolvimento do estudo e os resultados obtidos são detalhados no decorrer deste trabalho.


\textbf{Palavras-chave}: estruturas aeroespaciais. otimização. materiais compostos. parâmetros de laminação. painel reforçado. nastran sol 200.
\end{resumo}

% resumo em inglês
\begin{resumo}[Abstract]
\begin{otherlanguage*}{english}

The main objective of this Conclusion Graduation Work is to do a bibliographic revision about composite materials and optimization, also to describe how was the proccess to realize an structural optimization of a reinforced panel using the lamination parameters. It is known that the use of composite materials in aeronautic industry is increasing in last decades. Also, it is known that primary structures, as wings and empenages, has the tendency to be designed using reinforced panels of composite materials. It is due to the fact that structures in composite materials has higher values of specific resistence and specific rigity. The properties of composite materials can be optimized using the lamination parameters, showing the potencial of usage of this material. So, it were developted optimization models of the reinforced panel using the software Nastran to compare the results between reinforced panel produced of metallic material and reinforced panel produced of composite material. The development of this study and the results are detailed in this work.

\textbf{Keywords}: aerospace strucutures. optimization. composite materials. lamination parameters. reinforced panels. nastran sol 200.
\end{otherlanguage*}
\end{resumo}
%
% % resumo em francês
% \begin{resumo}[Résumé]
%  \begin{otherlanguage*}{french}
%     - à faire -
%
%    \textbf{Mots-clés}: aérospatiale. missile. simulation. hardware-in-the-loop.
%    model based design. model based system engineering. test driven development.
%  \end{otherlanguage*}
% \end{resumo}
%
% % resumo em espanhol
% \begin{resumo}[Resumen]
%  \begin{otherlanguage*}{spanish}
%    - a hacer -
%
%    \textbf{Palabras clave}: aeroespacial. misil. simulación.
%    hardware-in-the-loop. model based design. model based system engineering.
%    test driven development.
%  \end{otherlanguage*}
% \end{resumo}
