\chapter{Objetivos}

Como trabalho final do curso de graduação em Engenharia Aeroespacial na Universidade Federal de Minas Gerais, este trabalho de pesquisa foi realizado no âmbito de uma otimização estrutural de um painel reforçado utilizando como variáveis de otimização os parâmetros de laminação de cada componente estrutural.
Visando otimizar a estrutura de um painel reforçado quando submetido à uma determinada carga de compressão, será feito um modelo em elementos finitos utilizando o \emph{software FEMAP}. O revetimento do painel e os reforçadores serão otimizada utilizando a SOL 200 do \emph{Nastran} e análises de flambagem, SOL 105, serão realizadas visando obter os modos de flambagem da estrutura.

O objetivo inicial da otimização é obter as propriedades de um laminado que suporte a maior carga de flambagem, mantendo as espessuras do revestimento e a geometria e espessuras dos reforçadores constantes, portanto, variando somente as propriedades do laminado. Os parâmentros de laminação serão utilizados como variáveis de projeto e serão os resultados obtidos da otimização.

O outro objetivo do trabalho é criar um banco de dados para um determinado valor de espessura do laminado. Com este banco de dados de laminados será possível obter a sequência e os ângulos de cada lâmina do laminado com base nos valores de propriedades do laminado que foram otimizados.
