\chapter{Objetivos}

Como trabalho final do curso de graduação em Engenharia Aeroespacial na Universidade Federal de Minas Gerais, este trabalho de pesquisa foi realizado no âmbito de uma otimização estrutural de um painel reforçado utilizando como variáveis de otimização os parâmetros de laminação de cada componente estrutural.
Visando otimizar a estrutura de um painel reforçado quando submetido à uma determinada carga de compressão, será feito um modelo em elementos finitos utilizando o \emph{software FEMAP}. O revetimento do painel e os reforçadores serão otimizada utilizando a SOL 200 do \emph{Nastran} e análises de flambagem, SOL 105, serão realizadas visando obter os modos de flambagem da estrutura.

O objetivo inicial da otimização é obter uma estrutura otimizada do painel reforçado, ou seja, uma estrutura que suporte a carga aplicada mas que possua o menor peso possível. Para isso, as propriedades de um laminado, os parâmetros de laminação, e as espessuras serão utilizados como variáveis de projeto e serão os resultados obtidos da otimização.
