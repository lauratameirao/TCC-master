\chapter[Resultados]{Resultados}

\section{Análise teórica vs. Otimização do painel reforçado}
Esta seção está dividida em três subseções:
\begin{itemize}
\item Resultados da análise teórica do painel reforçado;
\item Resultados da otimização do painel reforçado fabricado em material metálico;
\item Comparação dos dois itens ateriores .
\end{itemize}

\subsection{Análise teórica do painel reforçado}
A análise teórica do painel reforçado seguiu a metodologia do \emph{Fator de Eficiência de Farrar} proposta por \cite{niu1997airframe}. O painel reforçado foi submetido a uma carga de compressão de intensidade 10000daN (22480lbs), obtendo-se portando uma carga de compressão linear (N) como segue:\\~\\

\centerline{N = 833,3 N/mm = 4758,5 lbs/in}\


Utilizou-se os valores otimizados na análise teórica:
F=0.81; $R_t$=2.25; $R_b$=0.65;
Da \autoref{fig_plotFarrar}, encontrou-se o valor de "f" correspondente a carga por largura (4758,5 lbs/in), utilizando F=0.81 para o Al 2024-T3 extrudado.\\~\\


\centerline{f = 41000 psi}\

Encontrou-se o módulo tangente $E_t$ correspondente ao valor de "f", da curva de módulo tangente do material, conforme \autoref{fig_tangentmodulus}.\\~\\

\centerline{$E_t$ = 5.0x$10^6$ psi}
\

Conforme \autoref{Farrar_Efficiency_t} e \autoref{Farrar_Efficiency_tw}, determinou-se os valores de $t$ e $t_w$.\\~\\

\centerline{$t = 0.501({\dfrac{NL}{E_t}})^{0.5} = 0.06 in = 1.53 mm$}\

\centerline{$t_w = 2.25t = 0.136in = 3.46 mm$}\


\subsection{Otimização do painel reforçado em material metálico}

\subsection{Comparação dos resultados}

\section{Análise teórica vs. Otimização do painel reforçado}
Esta seção está dividida em três subseções:
\begin{itemize}
\item Resultados da otimização do painel reforçado fabricado em material metálico;
\item Resultados da otimização do painel reforçado fabricado em material composto;
\item Comparação dos dois itens ateriores .
\end{itemize}

\subsection{Otimização do painel reforçado em material metálico}

\subsection{Otimização do painel reforçado em material composto}

\subsection{Comparação dos resultados}
