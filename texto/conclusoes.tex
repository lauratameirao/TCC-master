\chapter[Conclusão]{Conclusão}

A realização desse trabalhou permitiu que fosse realizado um estudo a respeito da otimização estrutural de um painel reforçado utilizando o \emph{solver Nastran} e uma solução de flambagem.

A utilização dos parâmetros de laminação permitiu que fosse realizada uma otimização consistente do painel reforçado em material composto e evitou que fosse feita a otimização discreta com base no número e direção de cada lâmina do laminado.

Por meio da otimização realizada, percebeu-se que para o tipo de problema abordado, o painel reforçado constiuído de material composto apresentava uma vantagem de massa estrutural em relação a uma painel constituído de material metálico.

Como sugestões para trabalhos futuros que deem continuidade ao tema abordado, citam-se a seguintes propostas:

\begin{itemize}
\item Otimização de outras estruturas utilizando o mesmo \emph{solver};
\item Otimização do painel reforçado em material composto submetido a outros tipos de carregamento além da força de compressão, por exemplo, carregamentos cisalhantes também. Com o objetivo de ver o comportamento do otimizador quando aplicado outros carregamentos na estrutura;
\item Desenvolvimento de um banco de laminados, repeitando as regras práticas de projetos utilizadas na indústria aeronáutica. Com isso é possível obter um laminado que possua as propriedades (parâmetros de laminação) que saíram como resultados da otimização.
\item Realizar a otimização do painel reforçado variando outros parâmetros além da espessura e dos parâmetros de laminação. Podem ser variados por exemplo, a largura e o espaçamento entre os reforçadores e a altura do reforçador. Para isso, deve-se estudar outras ferramentas disponíveis dentro da SOL 200 do \emph{Nastran}.
\end{itemize}
