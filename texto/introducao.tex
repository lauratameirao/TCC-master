\chapter[Introdução]{Introdução}

A utilização de materiais compostos em estruturas primárias tem aumentado gradualmente nas últimas décadas. Atualmente, no setor aeronáutico, estruturas primárias como asas, fuselagens e empenagens possuem a tendência de serem projetadas utilizando painéis reforçados constituídos de material composto. Isto se deve ao fato de as estruturas em materiais compostos possuírem elevadas resistência e rigidez específicas \cite{herencia2007optimization}. Além disso, variando-se a sequência do laminado e os ângulos de laminação, as propriedades do material composto podem ser otimizadas em vista do componente estrutural no qual o laminado será aplicado, apresentando um potencial de uso bastante elevado.

No decorrer dos anos, diversas técnicas de otimização foram desenvolvidas para auxiliar nos processos de obtenção do laminado ótimo para cada uso. Algumas das técnicas de otimização dos materiais compostos envolvem a variação do número de camadas do laminado e dos ângulos de laminação, e assumem que o material possui propriedades ortotrópicas, conforme utilizado por \cite{schmit1973optimum}. No entanto, segundo \cite{chamis1969buckling} pelo fato de os materiais compostos poderem apresentar características anisotrópicas, resultados não conservativos podem ser obtidos, durante otimizações nas quais o comportamento em flambagem é observado, caso a anisotropia flexural dos materiais não sejam consideradas. A otimização do número de camadas e dos ângulos de laminação de cada camada demanda um elevado custo computacional e consiste em um processo de otimização não linear com variáveis discretas e que possui um espaço de projeto não convexo.

Visando solucionar o problema de otimização das variáveis discretas da sequência de laminação dos materias compostos, \cite{miki1991optimum} propôs a utilização dos parâmetros de laminação. O método proposto por \cite{miki1991optimum}, considera que a rigidez no plano e a rigidez flexural de materiais compostos que possuem laminados simétricos e ortotrópicos são funções dos parâmetros de laminação, e esses parâmetros dependem da sequência de laminação. Com isso, os parâmetros de laminação podem ser utilizados como as variáveis de projeto durante a otimização e pontos ótimos de projeto podem ser obtidos em função desses parâmetros e da função objetivo.

O objetivo deste trabalho de conclusão de curso, é portanto, descrever um processo de otimização de um painel reforçado em material composto utilizando os parâmetros de laminção e as espessuras do revestimento e do reforçador como variáveis de projeto. O problema foi divido em duas etapas, na qual a primeira consistiu em validar o otimizador utilizado. Para isso, comparou-se um modelo de um painel reforçado em material metálico com um painel reforçado proposto por \cite{niu1997airframe}. Após a validação do otimizador, fez-se a otimização do painel reforçado em material composto utilizando os parâmetros de laminação e aplicando restrições de projetos presentes na indústria aeronáutica. A função objetivo da otimização é obter uma estrutura mais leve e que suporte a carga de compressão aplicada, variando as propriedades do material e as espessuras do revestimento e do reforçador. Nesta etapa, assumiu-se que os laminados dos reforçadores e do revestimento eram simétricos e ortotrópicos.

%\section{Desenvolvimento histórico}

%\subsection{\emph{Model Driven Architecture}}

%\begin{enumerate}
%  \item \emph{Computational Independent Model} -- CIM:\\
%A primeira etapa do projeto se refere às especificações %do código, escritas em linguagem humana.
%\end{enumerate}
