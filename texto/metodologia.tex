 \chapter[Metodologia]{Metodologia}

 \section{Desenvolvimento geral}
Visando realizar a otimização estrutural de um painel reforçado utilizando os parâmetros de laminação, realizou-se uma sequência de passos conformo segue:
\begin{enumerate}
  \item Análise teórica de um painel reforçado de material metálico utilizando a metodologia do \emph{Fator de Eficiência de Farrar} proposta por \cite{niu1997airframe}
  \item Otimização de um painel reforçado de material metálico utilizando o \emph{MSC Nastran} e comparação dos resultados obtidos com o resultado teórico obtido no item anterior. Nesta etapa visou-se obter a validação o método de otimização utilizado para assim poder aplicá-lo em outros paineis reforçados;
  \item Otimização de um painel reforçado de material composto utilizando o \emph{MSC Nastran} e comparação dos resultados obtidos com os resultados obtidos para um painel reforçado de material metálico.
\end {enumerate}

As seções seguintes descrevem a metodologia utilizada com mais detalhes e apresentam os modelos das otimizações realizadas e o capítulo seguinte contém os resultados obtidos.

\section{Análise teórica de um painel reforçado}
Visando realizar a análise teórica de um painel reforçado fabricado em material metálico, assume-se algumas proposições como segue e considera-se um painel conforme mostrado na \autoref{fig_StiffenedPanels}.

\begin{itemize}
\item Painel suficientemente largo para permitir que seja tratado como uma simples coluna, ou seja, não há restrições impostas nas bordas longitudinais do painel;
\item O painel possui uma fixação de apoio no final da sua estrutura, em relação ao comprimento L. Deve-se levar em consideração este comprimento como sendo o o comprimento efetivo do painel, e não o comprimento da báia;
\item As nervuras não colocam restrições à flambagem.
\end{itemize}

\begin{figure}[h]
	\caption{\label{fig_StiffenedPanels}Típico painel reforçado.}
  \centering
  \includegraphics[scale=1.0]{figura/StiffenedPanel}
	\legend{Fonte: \cite{niu1997airframe}}
\end{figure}
\

Nota-se que\

$t$ -  Espessura do revestimento \

$b_w$ - Altura do reforçador \

$b$ - Largura entre reforçadores \

$L$ - Comprimento efetivo \

A \autoref{fig_InitialBuckling} mostra a tensão de flambagem inicial de diversas combinações de paineis reforçados, como razão entre $\dfrac{f_i}{f_o}$, considerando a seguinte notação.

\begin{figure}[h]
	\caption{\label{fig_InitialBuckling}Tensão inicial de flambagem.}
  \centering
  \includegraphics[scale=1.0]{figura/InitialBuckling}
	\legend{Fonte: \cite{niu1997airframe}}
\end{figure}
\
$f$ - Tensão aplicada\

$f_E$ - Tensão de instabilidade de Euler\

$f_i$ - Tensão inicial de flambagem do painel\

$f_o$ - Tensão inicial de flambagem de um painel reforçado longo, com espaçamento entre reforçadores (b) e espessura do revestimento (t), simplesmente apoiado ao longo da borda = 3.62$E_t$$(\dfrac{t}{b})^2$\

Segundo \cite{niu1997airframe}, para um dado material, no qual a relação entre $f$ e $E_t$ é conhecida, o \emph{Fator de Eficiência de Farrar} pode ser dado por:

\begin{equation} \label{Farrar}
F = f(\dfrac{L}{N E_t})^{0.5}
\end{equation}
Onde

$N$ - Carga por unidade de comprimento do painel\

$E_t$ - Módulo tangente

Este fator pode ser utilizado, conforme \autoref{fig_plotFarrar} que mostra para um painel reforçado de Al 2024-T3.

\begin{figure}[h]
	\caption{\label{fig_plotFarrar}Tensão no painel \emph{vs.} índice estrutural $\dfrac{N}{L}$ para material Al 2024-T3.}
  \centering
  \includegraphics[scale=1.0]{figura/PlotFarrar}
	\legend{Fonte: \cite{niu1997airframe}}
\end{figure}
\

Sabe-se que a tensão incial de flambagem ($f_i$) é dada por:

\begin{equation} \label{InitialBuck}
f_i = (\dfrac{f_i}{f_o})[{3.62E_t(\dfrac{t}{b})^2}]
\end{equation}

E que a tensão de instabilidade de Euler ($f_E$) é dada por:

\begin{equation} \label{InitialEuler}
f_E = {\pi^2}E_t(\dfrac{\rho}{L})^2
\end{equation}

Onde $\rho$ corresponde ao raio de giração do painel reforçado em relação ao seu eixo neutro e tem que:

\begin{equation} \label{giracao}
{\rho^2} = {\dfrac{b^2{R_b}^3R_t}{12(1+R_bR_t)}}(4+R_bR_t)^2
\end{equation}

Relacionando a tensão no painel com a intensidade de carga tem-se:

\begin{equation} \label{StressPanel}
f = \dfrac{N}{t(1+R_bR_t)}
\end{equation}

Portanto, impõe-se a condição de que tensão aplicada é igual à tensão inicial de flambagem do painel que é igual à tensão de instabilidade de Euler ($f=f_i=f_E$).

Tomando-se a \autoref{InitialBuck} x \autoref{InitialEuler} x {\autoref{StressPanel}$^2$, obtem-se:

\begin{equation} \label{StressPanel2}
f^4 = \pi^2{E_t}^2(\frac{3.62{\rho}^2f_i}{f_ob^2L^2})[\dfrac{N^2}{(1+R_bR_t)^2}]
\end{equation}

Tirando a raiz quarta de ambos os lados, tem-se:

\begin{equation} \label{StressPanel3}
f = F(\dfrac{NE_t}{L})^{0.5}
\end{equation}

Onde o \emph{Fator de Eficiência de Farrar} vale:

\begin{equation} \label{Farrar_Efficiency}
F = 1.314{\dfrac{{R_b}^3R_t(4+R_bR_t)^{0.25}}{(1+R_bR_t)}}({\frac{f_i}{f_o}})^{0.25}
\end{equation}

\subsection{Limitação de projeto - Espaçamento entre reforçadores}
Considerando um espaçamento entre reforçadores (b) já previamente determinado, devido ao modelo em elementos finitos que foi desenvolvido para as etapas de otimização, tem-se, portanto, uma limitação de projeto. Segundo \cite{niu1997airframe}, para manter o valor de largura entre reforçadores constante (b) e o valor de espessura do reforçador também constante ($t_w$), tem-se as seguintes equações, que também são plotadas na \autoref{fig_plotstiffener}.

No caso deste estudo, considerou-se somente que a largura entre os reforçadores será constante, visando obter os outros parâmetros geométricos do painel reforçado ótimo.

\begin{figure}[ht]
	\caption{\label{fig_plotstiffener}Curvas para projetos com limitações - Espessura do reforçador ($t_w$) e largura do reforçador (b).}
  \centering
  \includegraphics[scale=1.0]{figura/PlotStiffener}
	\legend{Fonte: \cite{niu1997airframe}}
\end{figure}
\

Visando, portanto, projetar um painel reforçado otimizado, seguindo a metodologia de \emph{Fator de Eficiência de Farrar} proposta por \cite{niu1997airframe}, submetido a uma carga de intensidade N, alguns passos foram seguidos.

\begin{itemize}
\item Utilização de valores otimizados \

F=0.81; $R_t$=2.25; $R_b$=0.65z\

\end{itemize}
