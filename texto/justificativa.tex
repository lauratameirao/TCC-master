\chapter{Justificativa}

A computação vem tomando um espaço cada vez maior nos processos industriais.
Tarefas antes dadas apenas a humanos podem ser realizadas atualmente por programas especializados de forma muito mais rápida e eficiente.
Na engenharia, cada vez mais etapas da concepção de um produto (desde a definição de parâmetros iniciais até sua fabricação em massa) são realizadas por computadores.

Ainda assim, o desenvolvimento de \emph{softwares} permanece uma tarefa complicada e direcionada apenas a especialistas.
Ademais, programas mais complexos muitas vezes não são inteligíveis exceto para seu criador.
Tais fenômenos dificultam o uso da ferramenta computacional, que poderia facilitar muito o andamento dos processos, mas em vez disso cria empecilhos que diminuem a produtividade.

É com base nessa problemática que metodologias bem definidas de projeto de \emph{software} podem ser úteis.
A Engenharia de Sistemas Baseada em Modelos (MBSE), por exemplo, fornece ferramentas de análise de programas que independem de implementações.
Isso significa que, ao trabalhar com os modelos propostos pela MBSE, os projetistas não têm de se preocupar com minúcias de código e erros técnicos, como dificuldades de compilação, incompatibilidades de sistemas operacionais, falta de drivers, dentre outros empecilhos que desviam a atenção da lógica do programa.
A partir dos modelos desenvolvidos, pode-se inclusive gerar código automaticamente para uma plataforma desejada.
Essa técnica pode, portanto, eximir do projetista conhecimentos mais aprofundados de programação.

O desenvolvimento dirigido pelos testes, por sua vez, promove uma validação mais abrangente do código.
Uma vez definida a metodologia para aplicação dos testes, os códigos podem ser direcionados mais diretamente para o cumprimento de requisitos.
A análise dos códigos é, então, facilitada, o que ajuda a garantir sua robustez e evitar falhas.
No setor aeroespacial, falhas custam bastante dinheiro e, por vezes, vidas.
A situação é mais crítica ainda quando se trata do setor de defesa, o que justifica o grande interesse em códigos robustos.
