%% UNIVERSIDADE FEDERAL DE MINAS GERAIS
%% Engenharia Aeroespacial
%% Trabalho de Conclusão de Curso
%% Autor: Laura Tameirao Sampaio Rodrigues
%% Matrícula: 2013025569
%% Data: 20/07/2018

%% Baseado em:
% ------------------------------------------------------------------------
% ------------------------------------------------------------------------
% abnTeX2: Modelo de Trabalho Academico (tese de doutorado, dissertacao de
% mestrado e trabalhos monograficos em geral) em conformidade com
% ABNT NBR 14724:2011: Informacao e documentacao - Trabalhos academicos -
% Apresentacao
% ------------------------------------------------------------------------
% ------------------------------------------------------------------------

\documentclass[
	% -- opções da classe memoir --
	12pt,				  % tamanho da fonte
	openright,		% capítulos começam em pág ímpar (insere página vazia caso preciso)
	twoside,			% para impressão em recto e verso. Oposto a oneside
	a4paper,			% tamanho do papel.
	% -- opções da classe abntex2 --
	%chapter=TITLE,		% títulos de capítulos convertidos em letras maiúsculas
	%section=TITLE,		% títulos de seções convertidos em letras maiúsculas
	%subsection=TITLE,	% títulos de subseções convertidos em letras maiúsculas
	%subsubsection=TITLE,% títulos de subsubseções convertidos em letras maiúsculas
	% -- opções do pacote babel --
	english,			% idioma adicional para hifenização
	french,				% idioma adicional para hifenização
	spanish,			% idioma adicional para hifenização
	brazil				% o último idioma é o principal do documento
	]{abntex2}

% ---
% Pacotes básicos
% ---
\usepackage{lmodern}			    % Usa a fonte Latin Modern
\usepackage[T1]{fontenc}		  % Selecao de codigos de fonte.
\usepackage{amsmath}
\usepackage[utf8]{inputenc}		% Codificacao do documento (conversão automática dos acentos)
\usepackage{lastpage}			    % Usado pela Ficha catalográfica
\usepackage{indentfirst}		  % Indenta o primeiro parágrafo de cada seção.
\usepackage{color}				    % Controle das cores
\usepackage{graphicx}			    % Inclusão de gráficos
\usepackage{microtype} 			  % para melhorias de justificação
% ---

% % ---
% % Pacotes adicionais, usados apenas no âmbito do Modelo Canônico do abnteX2
% % ---
% \usepackage{lipsum}				% para geração de dummy text
% % ---

% ---
% Pacotes de citações
% ---
\usepackage[brazilian,hyperpageref]{backref}	 % Paginas com as citações na bibl
\usepackage[alf]{abntex2cite}	% Citações padrão ABNT

% ---
% CONFIGURAÇÕES DE PACOTES
% ---

% ---
% Configurações do pacote backref
% Usado sem a opção hyperpageref de backref
\renewcommand{\backrefpagesname}{Citado na(s) página(s):~}
% Texto padrão antes do número das páginas
\renewcommand{\backref}{}
% Define os textos da citação
\renewcommand*{\backrefalt}[4]{
	\ifcase #1 %
		Nenhuma citação no texto.%
	\or
		Citado na página #2.%
	\else
		Citado #1 vezes nas páginas #2.%
	\fi}%
% ---

% ---
% Informações de dados para CAPA e FOLHA DE ROSTO
% ---
\titulo{Otimização estrutural de um painel reforçado utilizando os parâmetros de laminação}
\autor{Laura Tameirao Sampaio Rodrigues}
\local{Belo Horizonte, MG - Brasil}
\data{\today}
\orientador{Prof. Rodrigo de Sá Martins}
\coorientador{-}
\instituicao{%
  Universidade Federal de Minas Gerais -- UFMG
  \par
  Escola de Engenharia
  \par
  Engenharia Aeroespacial}
\tipotrabalho{Trabalho de Conclusão de Curso}
% O preambulo deve conter o tipo do trabalho, o objetivo,
% o nome da instituição e a área de concentração
\preambulo{Trabalho de conclusão de curso de Engenharia Aeroespacial na Universidade Federal de Minas Gerais, centrado na otimização de uma estrutura em material composto utilizando os parâmetros de laminação}
% ---


% ---
% Configurações de aparência do PDF final

% alterando o aspecto da cor azul
\definecolor{blue}{RGB}{41,5,195}

% informações do PDF
\makeatletter
\hypersetup{
     	%pagebackref=true,
		pdftitle={\@title},
		pdfauthor={\@author},
    	pdfsubject={\imprimirpreambulo},
	    pdfcreator={LaTeX with abnTeX2},
		pdfkeywords={abnt}{latex}{abntex}{abntex2}{trabalho acadêmico},
		colorlinks=true,       		% false: boxed links; true: colored links
    	linkcolor=blue,          	% color of internal links
    	citecolor=blue,        		% color of links to bibliography
    	filecolor=magenta,      		% color of file links
		urlcolor=blue,
		bookmarksdepth=4
}
\makeatother
% ---

% ---
% Espaçamentos entre linhas e parágrafos
% ---

% O tamanho do parágrafo é dado por:
\setlength{\parindent}{1.3cm}

% Controle do espaçamento entre um parágrafo e outro:
\setlength{\parskip}{0.2cm}  % tente também \onelineskip

% ---
% compila o indice
% ---
\makeindex
% ---

% ----
% Início do documento
% ----
\begin{document}

% Seleciona o idioma do documento (conforme pacotes do babel)
%\selectlanguage{english}
\selectlanguage{brazil}

% Retira espaço extra obsoleto entre as frases.
\frenchspacing

% ----------------------------------------------------------
% ELEMENTOS PRÉ-TEXTUAIS
% ----------------------------------------------------------
% \pretextual

% ---
% Capa
% ---
\imprimircapa
% ---

% ---
% Folha de rosto
% (o * indica que haverá a ficha bibliográfica)
% ---
\imprimirfolhaderosto*
% ---

% ---
% Inserir a ficha bibliografica
% ---

% Isto é um exemplo de Ficha Catalográfica, ou ``Dados internacionais de
% catalogação-na-publicação''. Você pode utilizar este modelo como referência.
% Porém, provavelmente a biblioteca da sua universidade lhe fornecerá um PDF
% com a ficha catalográfica definitiva após a defesa do trabalho. Quando estiver
% com o documento, salve-o como PDF no diretório do seu projeto e substitua todo
% o conteúdo de implementação deste arquivo pelo comando abaixo:
%
% \begin{fichacatalografica}
%     \includepdf{fig_ficha_catalografica.pdf}
% \end{fichacatalografica}

\begin{fichacatalografica}
	\sffamily
	\vspace*{\fill}					% Posição vertical
	\begin{center}					% Minipage Centralizado
	\fbox{\begin{minipage}[c][8cm]{13.5cm}		% Largura
	\small
	\imprimirautor
	%Sobrenome, Nome do autor

	\hspace{0.5cm} \imprimirtitulo  / \imprimirautor. --
	\imprimirlocal, \imprimirdata-

	\hspace{0.5cm} \pageref{LastPage} p. : il. (algumas color.) ; 30 cm.\\

	\hspace{0.5cm} \imprimirorientadorRotulo~\imprimirorientador

	\hspace{0.5cm} \imprimircoorientadorRotulo~\imprimircoorientador\\

	\hspace{0.5cm}
	\parbox[t]{0.9\textwidth}{\imprimirtipotrabalho~--~\imprimirinstituicao,
	\imprimirdata.}\\

	\hspace{0.5cm}
		1. Otimização.
		2. Materiais compostos.
		2. Parâmetros de laminação.
		I. \imprimirorientador.
		II. Universidade Federal de Minas Gerais.
		III. Escola de Engenharia.
		IV. \imprimirtitulo
	\end{minipage}}
	\end{center}
\end{fichacatalografica}
% ---

% % ---
% % Inserir errata
% % ---
% \begin{errata}
% Elemento opcional da \citeonline[4.2.1.2]{NBR14724:2011}. Exemplo:
%
% \vspace{\onelineskip}
%
% FERRIGNO, C. R. A. \textbf{Tratamento de neoplasias ósseas apendiculares com
% reimplantação de enxerto ósseo autólogo autoclavado associado ao plasma
% rico em plaquetas}: estudo crítico na cirurgia de preservação de membro em
% cães. 2011. 128 f. Tese (Livre-Docência) - Faculdade de Medicina Veterinária e
% Zootecnia, Universidade de São Paulo, São Paulo, 2011.
%
% \begin{table}[htb]
% \center
% \footnotesize
% \begin{tabular}{|p{1.4cm}|p{1cm}|p{3cm}|p{3cm}|}
%   \hline
%    \textbf{Folha} & \textbf{Linha}  & \textbf{Onde se lê}  & \textbf{Leia-se}  \\
%     \hline
%     1 & 10 & auto-conclavo & autoconclavo\\
%    \hline
% \end{tabular}
% \end{table}
%
% \end{errata}
% % ---

% ---
% Inserir folha de aprovação
% ---

% Isto é um exemplo de Folha de aprovação, elemento obrigatório da NBR
% 14724/2011 (seção 4.2.1.3). Você pode utilizar este modelo até a aprovação
% do trabalho. Após isso, substitua todo o conteúdo deste arquivo por uma
% imagem da página assinada pela banca com o comando abaixo:
%
% \includepdf{folhadeaprovacao_final.pdf}
%
\begin{folhadeaprovacao}

  \begin{center}
    {\ABNTEXchapterfont\large\imprimirautor}

    \vspace*{\fill}\vspace*{\fill}
    \begin{center}
      \ABNTEXchapterfont\bfseries\Large\imprimirtitulo
    \end{center}
    \vspace*{\fill}

    \hspace{.45\textwidth}
    \begin{minipage}{.5\textwidth}
        \imprimirpreambulo
    \end{minipage}%
    \vspace*{\fill}
   \end{center}

   Trabalho aprovado. \imprimirlocal, \today:

   \assinatura{\textbf{\imprimirorientador} \\ Orientador}
   \assinatura{\textbf{\imprimircoorientador} \\ Coorientador}
   \assinatura{\textbf{Nome} \\ Convidado 1}
   %\assinatura{\textbf{Professor} \\ Convidado 3}
   %\assinatura{\textbf{Professor} \\ Convidado 4}

   \begin{center}
    \vspace*{0.5cm}
    {\large\imprimirlocal}
    \par
    {\large\imprimirdata}
    \vspace*{1cm}
  \end{center}

\end{folhadeaprovacao}
% ---

% ---
% Dedicatória
% ---
\begin{dedicatoria}
   \vspace*{\fill}
   \centering
   \noindent
   \textit{Dedicatória: a fazer.} \vspace*{\fill}
\end{dedicatoria}
% ---

% ---
% Agradecimentos
% ---
\begin{agradecimentos}

Aos meus pais, Amilvana e Marcos, por me permitirem e me apoiarem a estudar todos esses anos na UFMG, e por todo o suporte que eles sempre me deram nas minhas escolhas profissionais.

Ao meu irmão, Victor, pela amizade e apoio durante todo este tempo.

Aos meus colegas de faculdade, especialmente, aos colegas de equipe do Aerodesign, por todo o companherismo.

Aos meus colegas e supervisores do estágio, com quem tive a oportunidade de grande aprendizado, por todo o apoio e ensino durante esses últimos meses.

Aos todos os meus professores pelo ensinamento passado durante os últimos anos.

Ao meu orientador, Prof. Helio de Assis, e ao Prof. Rodrigo Martins, pela ajuda e suporte na realização deste trabalho.

\end{agradecimentos}

% ---

% ---
% Epígrafe
% ---
\begin{epigrafe}
    \vspace*{\fill}
	\begin{flushright}
		\textit{
Epigrafe: a fazer.
%``I'm afraid that the following syllogism may be used by some in the future.
%\\~\\
%Turing believes machines think\\
%Turing lies with men\\
%Therefore machines do not think
%\\~\\
%Yours in distress,\\
%Alan''\\~\\
%(Alan Turing, Letter to Norman Routledge, 1952)
}
	\end{flushright}
\end{epigrafe}
% ---

% ---
% RESUMOS
% ---
% resumo em português
\setlength{\absparsep}{18pt} % ajusta o espaçamento dos parágrafos do resumo
\begin{resumo}
 % Segundo a \citeonline[3.1-3.2]{NBR6028:2003}, o resumo deve ressaltar o
 % objetivo, o método, os resultados e as conclusões do documento. A ordem e a extensão
 % destes itens dependem do tipo de resumo (informativo ou indicativo) e do
 % tratamento que cada item recebe no documento original. O resumo deve ser
 % precedido da referência do documento, com exceção do resumo inserido no
 % próprio documento. (\ldots) As palavras-chave devem figurar logo abaixo do
 % resumo, antecedidas da expressão Palavras-chave:, separadas entre si por
 % ponto e finalizadas também por ponto.

 O intuito deste Trabalho de Conclusão de Curso é o de realizar uma revisão bibliográfica a respeito dos temas de materiais compostos e otimização, bem como descrever como foi feita uma otimização estrutural de um painel reforçado utilizando os parâmetros de laminação. Sabe-se que a utilização dos materiais compostos em estruturas aeronáuticas tem aumentado gradualmente nas últimas décadas, e também, que estruturas primárias como asas e empenagens possuem a tendência de serem projetadas utilizando painéis reforçados constituídos de material composto. Isto se deve ao fato de as estruturas em materiais compostos possuírem elevados valores de resistência e rigidez específicas. As propriedades do material composto podem ser otimizadas utilizando como ferramenta os parâmetros de laminação, apresentando um potencial de uso bastante elevado para esse tipo de material. Portanto, foram desenvolvidos modelos de otimização do painel reforçado utilizando o \emph{software Nastran} para comparar os resultados entre painéis reforçados fabricados de material metálico e de material composto. O desenvolvimento do estudo e os resultados obtidos são detalhados no decorrer deste trabalho.


\textbf{Palavras-chave}: estruturas aeroespaciais. otimização. materiais compostos. parâmetros de laminação. painel reforçado. nastran sol 200.
\end{resumo}

% resumo em inglês
\begin{resumo}[Abstract]
\begin{otherlanguage*}{english}

The main objective of this Conclusion Graduation Work is to do a bibliographic revision about composite materials and optimization, also to describe how was the proccess to realize an structural optimization of a reinforced panel using the lamination parameters. It is known that the use of composite materials in aeronautic industry is increasing in last decades. Also, it is known that primary structures, as wings and empenages, has the tendency to be designed using reinforced panels of composite materials. It is due to the fact that structures in composite materials has higher values of specific resistence and specific rigity. The properties of composite materials can be optimized using the lamination parameters, showing the potencial of usage of this material. So, it were developted optimization models of the reinforced panel using the software Nastran to compare the results between reinforced panel produced of metallic material and reinforced panel produced of composite material. The development of this study and the results are detailed in this work.

\textbf{Keywords}: aerospace strucutures. optimization. composite materials. lamination parameters. reinforced panels. nastran sol 200.
\end{otherlanguage*}
\end{resumo}
%
% % resumo em francês
% \begin{resumo}[Résumé]
%  \begin{otherlanguage*}{french}
%     - à faire -
%
%    \textbf{Mots-clés}: aérospatiale. missile. simulation. hardware-in-the-loop.
%    model based design. model based system engineering. test driven development.
%  \end{otherlanguage*}
% \end{resumo}
%
% % resumo em espanhol
% \begin{resumo}[Resumen]
%  \begin{otherlanguage*}{spanish}
%    - a hacer -
%
%    \textbf{Palabras clave}: aeroespacial. misil. simulación.
%    hardware-in-the-loop. model based design. model based system engineering.
%    test driven development.
%  \end{otherlanguage*}
% \end{resumo}

% ---

% % ---
% % inserir lista de ilustrações
% % ---
% \pdfbookmark[0]{\listfigurename}{lof}
% \listoffigures*
% \cleardoublepage
% % ---
%
% % ---
% % inserir lista de tabelas
% % ---
% \pdfbookmark[0]{\listtablename}{lot}
% \listoftables*
% \cleardoublepage
% % ---

% ---
% inserir lista de abreviaturas e siglas
% ---

\begin{siglas}

  \item[PL] \emph{Parâmetros de laminação}
  \item[CLT] \emph{Classical Theory of Lamination}
  \item[FEM] \emph{Finite Element Model}


\end{siglas}

% ---
%
% % ---
% % inserir lista de símbolos
% % ---
% % 
% \begin{simbolos}
%   \item[$ \alpha $] teste-alfa
% \end{simbolos}

% % ---

% ---
% inserir o sumario
% ---
\pdfbookmark[0]{\contentsname}{toc}
\tableofcontents*
\cleardoublepage
% ---



% ----------------------------------------------------------
% ELEMENTOS TEXTUAIS
% ----------------------------------------------------------
\textual

% ----------------------------------------------------------
% Objetivos
% ----------------------------------------------------------
\chapter{Objetivos}

Como trabalho final do curso de graduação em Engenharia Aeroespacial na Universidade Federal de Minas Gerais, este trabalho de pesquisa foi realizado com o objetivo de aplicar os conhecimentos adquiridos durante o curso, executando um trabalho de engenharia no âmbito de uma otimização estrutural de um painel reforçado utilizando como variáveis os parâmetros de laminação de cada componente estrutural e suas espessuras.
Visando otimizar a estrutura de um painel reforçado quando submetido à uma determinada carga de compressão, foi feito um modelo em elementos finitos utilizando o \emph{software FEMAP} e o \emph{solver Nastran}. O revestimento do painel e a espessura dos reforçadores serão otimizados utilizando a SOL 200 do \emph{Nastran} e, para as análises de flambagem, foi utilizado a SOL 105, obtendo os modos de flambagem da estrutura.


% ----------------------------------------------------------
% Justificativa
% ----------------------------------------------------------
%\chapter{Justificativa}

A computação vem tomando um espaço cada vez maior nos processos industriais.
Tarefas antes dadas apenas a humanos podem ser realizadas atualmente por programas especializados de forma muito mais rápida e eficiente.
Na engenharia, cada vez mais etapas da concepção de um produto (desde a definição de parâmetros iniciais até sua fabricação em massa) são realizadas por computadores.

Ainda assim, o desenvolvimento de \emph{softwares} permanece uma tarefa complicada e direcionada apenas a especialistas.
Ademais, programas mais complexos muitas vezes não são inteligíveis exceto para seu criador.
Tais fenômenos dificultam o uso da ferramenta computacional, que poderia facilitar muito o andamento dos processos, mas em vez disso cria empecilhos que diminuem a produtividade.

É com base nessa problemática que metodologias bem definidas de projeto de \emph{software} podem ser úteis.
A Engenharia de Sistemas Baseada em Modelos (MBSE), por exemplo, fornece ferramentas de análise de programas que independem de implementações.
Isso significa que, ao trabalhar com os modelos propostos pela MBSE, os projetistas não têm de se preocupar com minúcias de código e erros técnicos, como dificuldades de compilação, incompatibilidades de sistemas operacionais, falta de drivers, dentre outros empecilhos que desviam a atenção da lógica do programa.
A partir dos modelos desenvolvidos, pode-se inclusive gerar código automaticamente para uma plataforma desejada.
Essa técnica pode, portanto, eximir do projetista conhecimentos mais aprofundados de programação.

O desenvolvimento dirigido pelos testes, por sua vez, promove uma validação mais abrangente do código.
Uma vez definida a metodologia para aplicação dos testes, os códigos podem ser direcionados mais diretamente para o cumprimento de requisitos.
A análise dos códigos é, então, facilitada, o que ajuda a garantir sua robustez e evitar falhas.
No setor aeroespacial, falhas custam bastante dinheiro e, por vezes, vidas.
A situação é mais crítica ainda quando se trata do setor de defesa, o que justifica o grande interesse em códigos robustos.


% ----------------------------------------------------------
% Introdução (exemplo de capítulo sem numeração, mas presente no Sumário)
% ----------------------------------------------------------
\chapter[Introdução]{Introdução}

A utilização de materiais compostos em estruturas primárias tem aumentado gradualmente nas últimas décadas. Atualmente, no setor aeronáutico, estruturas primárias como asas, fuselagens e empenagens possuem a tendência de serem projetadas utilizando painéis reforçados constituídos de material composto. Isto se deve ao fato de as estruturas em materiais compostos possuírem elevadas resistência e rigidez específicas \cite{herencia2007optimization}. Além disso, variando-se a sequência do laminado e os ângulos de laminação, as propriedades do material composto podem ser otimizadas em vista do componente estrutural no qual o laminado será aplicado, apresentando um potencial de uso bastante elevado.

No decorrer dos anos, diversas técnicas de otimização foram desenvolvidas para auxiliar nos processos de obtenção do laminado ótimo para cada uso. Algumas das técnicas de otimização dos materiais compostos envolvem a variação do número de camadas do laminado e dos ângulos de laminação, e assumem que o material possui propriedades ortotrópicas, conforme utilizado por \cite{schmit1973optimum}. No entanto, segundo \cite{chamis1969buckling} pelo fato de os materiais compostos poderem apresentar características anisotrópicas, resultados não conservativos podem ser obtidos, durante otimizações nas quais o comportamento em flambagem é observado, caso a anisotropia flexural dos materiais não sejam consideradas. A otimização do número de camadas e dos ângulos de laminação de cada camada demanda um elevado custo computacional e consiste em um processo de otimização não linear com variáveis discretas e que possui um espaço de projeto não convexo.

Visando solucionar o problema de otimização das variáveis discretas da sequência de laminação dos materias compostos, \cite{miki1991optimum} propôs a utilização dos parâmetros de laminação. O método proposto por \cite{miki1991optimum}, considera que a rigidez no plano e a rigidez flexural de materiais compostos que possuem laminados simétricos e ortotrópicos são funções dos parâmetros de laminação, e esses parâmetros dependem da sequência de laminação. Com isso, os parâmetros de laminação podem ser utilizados como as variáveis de projeto durante a otimização e pontos ótimos de projeto podem ser obtidos em função desses parâmetros e da função objetivo.

O objetivo deste trabalho de conclusão de curso, é portanto, descrever um processo de otimização de um painel reforçado em material composto utilizando os parâmetros de laminção como variáveis de projeto. O problema será divido em duas etapas, na qual a primeira consistirá na otimização dos parâmetros de laminação aplicando restrições de projetos presentes na indústria aeronáutica. A função objetivo da otimização é obter uma estrutura que suporte a maior carga de flambagem variando somente as propriedades do material e mantendo a geometria dos reforçadores e do painel constantes. Nesta etapa, será assumido que os laminados dos reforçadores e do revestimento sejam simétricos e ortotrópicos. Após a obtenção dos parâmetros de laminação com a otimização estrutural, a sequência de laminação do laminado deverá ser obtida. Para isso, será criado um banco de dados de laminados com restrições e critérios de boas práticas e então os melhores laminados que atenderem aos critérios pré-estabelecidos serão selecionados. 

%\section{Desenvolvimento histórico}

%\subsection{\emph{Model Driven Architecture}}

%\begin{enumerate}
%  \item \emph{Computational Independent Model} -- CIM:\\
%A primeira etapa do projeto se refere às especificações %do código, escritas em linguagem humana.
%\end{enumerate}


% ----------------------------------------------------------
\chapter[Materiais Compostos]{Materiais Compostos}

\section{Desenvolvimento histórico}
A implementação do uso de materiais compostos na indústria aeronáutica civil e militar seguiu os estágios típicos da implementação de qualquer nova tecnologia no mercado. Segundo \cite{kassapoglou2013design}, primeiramente o uso da tecnologia de materiais compostos foi limitado às estruturas secundárias visto que minimizavam os riscos envolvidos e também possibilitava a coleta de dados, o que viabilizava uma melhor compreensão do comportamento das estruturas que possuiam essa tecnologia.

De acordo com \cite{daniel2006engineering}, em 1942 o primeiro barco constiuído de fibra de vidro foi fabricado, e nos anos 1950 as primeiras aplicações com materiais compostos em mísseis foram realizadas. Referindo-se a indústria aeronáutica no último século, o primeiro uso de materiais compósitos mais avançados, segundo \cite{kassapoglou2013design}, ocorreu no final da década de 1950 na aeronave \emph{Akaflieg Phonix FS-24},que foi um planador projetado por professores e alunos da Universidade de Stuttgart e construído, inicialmente de madeira balsa, e que posteriormente teve sua estrutura alterada para um sanduíche de compósitos de fibra de vidro com madeira tipo balsa. Após isso, no fim dos anos 1960, com a nova geração de materiais compósitos avançados, como o carbono, a indústria de helicópteros foi a primeira a utilizá-los em estruturas primárias, destacando o projeto do \emph{Aerospatiale AS-341 Gazelle}. Este helicóptero foi considerado um dos mais modernos na época, não só porque ele possuía um inovador rotor de cauda reduzindo drasticamente a emissão de ruídos, mas também, pelo fato de as pás do rotor principal serem constituídas de material composto.

Por volta dos anos 1970 as primeiras aeronaves majoritariamente constituídas de materiais compósitos começaram a surgir. Essas aeronaves eram pequenas e normalmente para uso recreativo ou para acrobacias, visto que com o uso de compósitos obtinham uma redução de peso estrutural e portanto, aeronaves mais rápidas e ágeis quando comparadas às aeronaves da época, e também pelo fato de os requisitos de certificação estrutural em materiais compostos para aeronaves menores serem mais facilmente cumpridos se comparados às aeronaves de grande porte. Além disso, de acordo com \cite{kassapoglou2013design}, a performance dos materias compostos não era completamente conhecida, por exemplo, a sensitividade desse tipo de material ao dano por impacto e suas implicações para o projeto só foram ser melhor conhecidas no final dos anos 1970. Portanto, somente no final dos anos 1970 e início dos anos 1980 que a utilização de materiais compostos começou a ser expandida para aeronaves de porte maior, como a concepção da empenagem horizontal do \emph{Boeing 737}, que era contruída de um sanduíche de materias compostos. Seguindo a aplicação em grande escala de materiais compostos, destaca-se o \emph{Airbus A320}, no qual tanto a empenagem horizontal e vertical, quanto as superfícies de controle foram projetadas e fabricadas utilizando material composto.

A proxíma aplicação significante desse tipo de material em estruturas primárias foi no início dos anos 1990 com o \emph{Boeing 777}, em que além das empenagens e superfícies de controle, as vigas principais do piso eram constituídas de material composto.
No entanto, segundo \cite{daniel2006engineering}, o maior sinal de aceitação do uso de materiais compostos na indústria aeronáutica civil, ocorreu no \emph{Boeing 787 Dreamliner}, em que materiais como carbono/epoxy e grafite/titânio constituíam cerca de 50\% do peso da aeronave, incluindo majoritariamente asas e fuselagem. Destaca-se também o \emph{Airbus A380}, que utiliza materiais compostos, incluindo o \emph{GLARE}, um laminado híbrido de fibra de vidro/epoxy/alumínio, que combina as vantagens e desvantagens dos materiais metálicos e compostos.

Observa-se, portanto, que o uso dos materiais compostos vem aumentando na indústria aeronáutica. Uma maneira de perceber o aumento do uso de materiais compostos nessa indústria é com base na \autoref{fig_usecomposites}, em que fica claro o aumento percentual da utilização desse tipo de material nas estruturas de vários modelos de aeronaves.

\begin{figure}[h]
	\caption{\label{fig_usecomposites}Uso de materiais compostos na indústria aeronáutica}
  \centering
  \includegraphics[scale=1.0]{figura/UseOfComposites}
	\legend{Fonte: \cite[p. 6]{kassapoglou2013design}}
\end{figure}

\section{Visão geral}
De acordo com \cite{daniel2006engineering}, os materiais compostos possuem diversas vantagens de utilização em relação aos materiais metálicos como elevada resistência, elevada rigidez, vida longa em fadiga, baixa densidade e alta adaptabilidade em relação a função de utilização pretendida pela estrutura. A superior perfomance estrutural dos materiais compostos se deve basicamente às elevadas resistência e rigidez específicas e à anisotropia do material, visto que devido à esta última característica, o material composto possui diversos graus de liberdade para uma configuração ótima do laminado. No geral, devido ao elevado número de graus de liberdade é possível realizar a otimização do laminado em material composto para diversas restrições de projeto e objetivos, como menor peso estrutural, máxima estabilidade dinâminca e/ou menor custo de fabricação. No entanto, todo o processo requer um confiável banco de dados das propriedades dos materiais, métodos de análises estruturais, técnicas de modelagem e simulações padronizadas e certificadas. Logo, devido às numerosas opções disponíveis, os processos e análises acabam se tornando mais complexas em relação aos materiais convencionais.

Os materiais compostos possuem algumas limitações de uso em relação ao materiais metálicos. Do ponto de vista da micromecânica, as fibras dos materiais compostos possuem uma grande variabilidade nas propriedades de resistência e concentradores de tensão locais reduzem consideravelmente a resistência a tração das estruturas projetadas em materiais compostos. Em relação a macromecânica, a anisotropia do material pode ser utilizada considerada como uma vantagem visto que o comportamento do material pode ser variado, no entando, esta mesma característica faz com que as análises desses materiais sejam muito mais complexas \cite{daniel2006engineering}.

\section{Parâmetros de laminação}


\[
\begin{bmatrix}
    \xi_{11}^2       & x_{12} & x_{13} & \dots & x_{1n} \\
    x_{21}       & x_{22} & x_{23} & \dots & x_{2n} \\
    \hdotsfor{5} \\
    x_{d1}       & x_{d2} & x_{d3} & \dots & x_{dn}
\end{bmatrix}
=
\begin{bmatrix}
    x_{11} & x_{12} & x_{13} & \dots  & x_{1n} \\
    x_{21} & x_{22} & x_{23} & \dots  & x_{2n} \\
    \vdots & \vdots & \vdots & \ddots & \vdots \\
    x_{d1} & x_{d2} & x_{d3} & \dots  & x_{dn}
\end{bmatrix}
\]

\section{Práticas de projeto para materiais compostos}
Esta seção apresenta regras e práticas relevantes utilizadas durante o projeto de estruturas em materiais compostos na indústria aeronáutica.

\subsection{Laminados simétricos}
Os laminados que possuem um sequência	de ângulos das lâminas simétrico em relação ao plano médio, são chamados de laminados simétricos. Conforme descrito por \cite{mil2002handbook} e \cite{niucomposite}, a maior vantagem da utilização de um laminado simétrico é o desacoplamento entre o comportamento de membrana e flexão da estrutura.

Em um laminado simétrico, conforme notação apresentada na \autoref{fig_laminate} e conforme a \autoref{eq_matrixB} matriz [B] do laminado se anula.

\begin{figure}[h]
	\caption{\label{fig_laminate}Notação para espessura do laminado e sequência das lâminas}
  \centering
  \includegraphics[scale=1.0]{figura/Laminate}
	\legend{Fonte: \cite{mil2002handbook}}
\end{figure}

\begin{equation} \label{eq_matrixB}
B_{ij}
=
\sum_{k=1}^N (\overline{Q}_{ij})_k [z_k^2 - (z_{k-1})^2]
\end{equation}

Sabe- se que $ \overline{Q}_{ij} $ corresponde a rigidez da lâmina. E sabe-se também que a matriz $ B_{ij} $ é a responsável pelo acomplamento entre a reposta no plano e a flexão do laminado. Portanto, conforme \cite{nasa1997guidelines}, quando a matriz $ B_{ij} $ não é zerada, um carregamento no plano induz curvaturas, e momentos de flexão induzem deformações no plano. Nota-se pela \autoref{eq_matrixB} que a matriz $ B_{ij} $ possui termos da coordenada z elevados ao quadrado, portanto, quando o laminado possui simetria geométrica e de materiais em relação ao plano médio, este termo é zerado.

\subsection{Laminados balanceados}
Os laminados balanceados são aqueles em que todas as lâminas, com exceção das de 0$^{\circ}$ e das de 90$^{\circ}$, devem ocorrer em pares de $ +\theta $ e $ -\theta $ acima e abaixo do plano médio do laminado. Para o conjunto de laminados compostos por lâminas com ângulos 0/$\pm$45/90, cada lâmina de +45$^{\circ}$ deve ser acompanhada de um lâmina de -45$^{\circ}$.
Laminados balanceados possuem vantagens similares às vantagens do laminados simétricos. Uma delas é que o acoplamento de membrana entre o comportamento normal e de cisalhamento no plano da estrutura é removido, visto que ambos os coeficientes, $ A_{16} $ e $ A_{26} $, são iguais a zero \cite{nasa1997guidelines}. Este comportamento pode ser explicado observado as equações do carregamento de membrana de um laminado simétrico, \autoref{eq_loadingA}, \autoref{eq_A16} e \autoref{eq_A26}.

\begin{equation} \label{eq_loadingA}
\begin{bmatrix}
    N_{x} \\
    N_{y} \\
    N_{xy} \\
\end{bmatrix}
=
\begin{bmatrix}
    A_{11} & A_{12} & A_{16}\\
    A_{12} & A_{22} & A_{26}\\
    A_{16} & A_{26} & A_{66}\\
\end{bmatrix}
\begin{bmatrix}
    \varepsilon_{x}^o \\
    \varepsilon_{y}^o \\
    \gamma_{xy}^o \\
\end{bmatrix}
\end{equation}

\begin{equation} \label{eq_A16}
A_{16}
=
\sum_{k=1}^N (\overline{Q}_{16})_k t_k
\end{equation}

\begin{equation} \label{eq_A26}
A_{26}
=
\sum_{k=1}^N (\overline{Q}_{26})_k t_k
\end{equation}

Onde

\begin{equation} \label{eq_Q16}
(\overline{Q}_{16})_k
=
({Q}_{11}-{Q}_{12}-2{Q}_{66})_k \sin\theta \cos^3\theta + ({Q}_{11}-{Q}_{22}-2{Q}_{66})_k \sin^3\theta \cos\theta
\end{equation}
\begin{equation} \label{eq_Q26}
(\overline{Q}_{26})_k
=
({Q}_{11}-{Q}_{12}-2{Q}_{66})_k \sin^3\theta \cos\theta + ({Q}_{11}-{Q}_{22}-2{Q}_{66})_k \sin\theta \cos^3\theta
\end{equation}

Sabe- se que $ \overline{Q}_{ij} $ corresponde a rigidez da lâmina e que $ t_k $ corresponde a espessura da lâmina. Nota-se também que ambas as expressões de $ A_{16} $ e $ A_{26} $ contém potências ímpares de $ \sin\theta $ e $ \cos\theta $. Logo lâminas com ângulos de 0$^{\circ}$ e 90$^{\circ}$ não contribuem para os termos de $ A_{16} $ e $ A_{26} $ e estes termos são reduzidos a zero em qualquer laminado balanceado \cite{nasa1997guidelines}.

A \autoref{fig_balancedlaminate} apresenta dois laminados, um desbalanceado, visto que faltam lâminas com -45$^{\circ}$ e um balanceado.

\begin{figure}[ht]
	\caption{\label{fig_balancedlaminate}Laminado desbalanceado e laminado balanceado}
  \centering
  \includegraphics[scale=1.0]{figura/BalancedLaminate}
	\legend{Fonte: \cite{mil2002handbook}}
\end{figure}

Portanto, satisfazendo esta prática de projeto de utilizar somente laminados balanceados, tem-se a seguinte \autoref{eq_result_loadingA} resultante para tensão-deformação

\begin{equation} \label{eq_result_loadingA}
\begin{bmatrix}
    N_{x} \\
    N_{y} \\
    N_{xy} \\
\end{bmatrix}
=
\begin{bmatrix}
    A_{11} & A_{12} & 0\\
    A_{12} & A_{22} & 0\\
    0 & 0 & A_{66}\\
\end{bmatrix}
\begin{bmatrix}
    \varepsilon_{x}^o \\
    \varepsilon_{y}^o \\
    \gamma_{xy}^o \\
\end{bmatrix}
\end{equation}

\subsection{Laminados balanceados}


% % ----------------------------------------------------------
% % Metodologia
% % ----------------------------------------------------------
% % \chapter{Metodologia}
%
% \section{Desenvolvimento geral}
%
% A criação do simulador em questão deve seguir algumas etapas de modo a adequá-lo aos requisitos iniciais.
% Essas etapas são baseadas na MDA, mas com algumas modificações para permitir o trabalho com o Simulink.
%
%   \subsection{CIM}
%
%   Em um primeiro momento, deve-se definir explicitamente quais as especificações dos códigos e modelos a serem criados.
%   Essa definição será usada como base para todo o projeto.
%
%   A princípio, espera-se que os requisitos sejam feitos em um \emph{software} específico para a Engenharia de Sistemas.
%   Entretanto, eles serão definidos aqui em linguagem natural (português), sem o formalismo técnico proporcionado pelo \emph{software}, mas com os devidos cuidados.
%   O resultado final não deve ser alterado, uma vez que o formalismo serviria apenas para agilizar o processo, mantendo a robustez das especificações.
%
%   \subsection{PIM}
%
%   O PIM deve conter todas as informações necessárias para a implementação do que se pretende modelar.
%   Nessa etapa do projeto, serão criados alguns diagramas UML de modo a visualizar com mais detalhes o que foi definido na etapa do CIM.
%
%   \begin{description}
%     \item[Diagrama de Componentes:]
%     Representa módulos do \emph{software} em questão.
%     Esses módulos podem possuir portas, implementar ou utilizar interfaces.
%     As interfaces contêm métodos bem definidos, e conectam diferentes componentes.
%     Cada componente pode ser composto por diferentes classes.
%     Este diagrama é bastante útil para identificar os módulos funcionais do programa.
%
%     \item[Diagrama de Implantação:]
%     Representa a estrutura física do programa, ou seja, onde cada parte será executada.
%     É possível modelar tanto o \emph{hardware} quanto os ambientes de execução dos programas, como por exemplo máquinas virtuais.
%     Também é possível criar conexões entre partes do modelo, e importar componentes para definir onde cada um atuará.
%     Este diagrama serve para ter-se uma ideia concreta da instalação de cada parte do \emph{software}.
%
%     \item[Diagrama de Classes:]
%     Quando se utiliza a programação orientada objeto, ideal para simulações, é possível representar cada classe (descrição de cada tipo de objeto) neste diagrama.
%     As classes são bastante detalhadas, contendo atributos e operações, além de relações com outras classes.
%     Isso permite inclusive a geração automática de código para este diagrama, que fornece uma descrição bastante completa da execução.
%   \end{description}
%
%   \subsection{Modelo Simulink}
%
%   Uma vez que a estrutura geral do código já está definida, fazem-se então os modelos Simulink necessários para a simulação.
%   Grande parte da simulação será aproveitada dos modelos criados em sala de aula no curso de CARLBERTO, na graduação em ITA.
%   Nessa etapa, deve-se prestar atenção especial à TDD.
%
%   \subsection{Códigos}
%
%   Enquanto se desenvolve o modelo Simulink, pretende-se escrever o mínimo possível de código.
%   Entretanto, todo código que deve ser escrito passará por testes definidos pela TDD.
%
%
% \section{Versões do simulador}
%
% Tendo em vista um desenvolvimento linear do simulador HITL, propõem-se algumas etapas anteriores, mais simples, mas que fazem parte do desenrolar do produto final.
% As etapas são listadas a seguir.
%
%   \subsection{Versão 0: Simulador SITL}
%
%   Após a conclusão das etapas do CIM e do PIM, um modelo Simulink é criado baseando-se nos dados gerados nos modelos.
%   Pretende-se utilizar o minimo possível de \emph{toolboxes} do Simulink, uma vez que cada \emph{toolbox} representa um custo adicional.
%
%   O simulador deve exibir dados que permitam visualizar o que acontece durante sua execução.
%   Ainda, é desejável a implementação de uma simulação \emph{soft real time}, ou seja, que simule uma situação de voo com um tempo de simulação igual ao tempo absoluto da situação simulada.
%   Não há a necessidade de analisar a fundo o tempo de execução de cada parte do programa, pois atrasos de execução não teriam implicações críticas para o resultado da simulação.
%
%   TESTES?
%
%   \subsection{Versão 1: Simulador em malha aberta}
%
%   A partir da simulação SITL e da experiência adquirida no seu processo de construção, pretende-se revisar o CIM e adaptá-lo para considerar uma simulação HITL em malha aberta.
%   Logo após, adapta-se o PIM e, posteriormente, o modelo Simulink para que sigam as novas especificações.
%
%   A comunicação com o dispositivo externo deve ser feita da maneira mais modular possível.
%   A modularidade permite que a alteração entre uma versão e outra seja facilitada.
%   Ela permite também que o sistema seja mais facilmente compreendido por terceiros.
%
%   O simulador será projetado para atuadores fornecidos pela empresa SIATT.
%   Ele será testado, portanto, com estes atuadores, que recebem e enviam mensagens pelo protocolo conhecido como Barramento CAN.
%   Apenas serão consideradas, para esta fase, as mensagens enviadas aos atuadores.
%
%   \subsection{Versão 2: Simulador em malha fechada}
%
%   Uma segunda revisão dos requisitos pode então ser feita nos mesmos moldes da primeira revisão.
%   Porém, esta versão considera também as mensagens que serão recebidas pelos atuadores.
%   Testes serão realizados a fim de analisar a estabilidade da malha.

%
% % ----------------------------------------------------------
% % Resultados
% % ----------------------------------------------------------
% \chapter[Resultados]{Resultados}

\section{Análise teórica vs. Otimização do painel reforçado}
Esta seção está dividida em três subseções:
\begin{itemize}
\item Resultados da análise teórica do painel reforçado;
\item Resultados da otimização do painel reforçado fabricado em material metálico;
\item Comparação dos dois itens ateriores .
\end{itemize}

\subsection{Análise teórica do painel reforçado}
A análise teórica do painel reforçado seguiu a metodologia do \emph{Fator de Eficiência de Farrar} proposta por \cite{niu1997airframe}. O painel reforçado foi submetido a uma carga de compressão de intensidade 10000daN (22480lbs), obtendo-se portando uma carga de compressão linear (N) como segue:\\~\\

\centerline{N = 833,3 N/mm = 4758,5 lbs/in}\


Utilizou-se os valores otimizados na análise teórica:
F=0.65; $R_t$=1.50; $R_b$=0.33;
Da \autoref{fig_plotFarrar}, encontrou-se o valor de "f" correspondente a carga por largura (4758,5 lbs/in), utilizando F=0.65 para o Al 2024-T3 extrudado.\\~\\


\centerline{f = 35000 psi}\

Encontrou-se o módulo tangente $E_t$, aproximado, correspondente ao valor de "f", da curva de módulo tangente do material, conforme \autoref{fig_tangentmodulus}.\\~\\

\centerline{$E_t$ = 8.5x$10^6$ psi}
\

Conforme \autoref{Farrar_Efficiency_t} e \autoref{Farrar_Efficiency_tw}, determinou-se os valores de $t$ e $t_w$.\\~\\

\centerline{$t = 0.501({\dfrac{NL}{E_t}})^{0.5} = 0.06 in = 1.524 mm$}\

\centerline{$t_w = 2.25t = 0.136in = 3.454 mm$}\


\subsection{Otimização do painel reforçado em material metálico}
Desenvolveu-se um modelo de otimização do reforçador em material metálico, no qual, o material possuia as mesmas propriedades de módulo de elasticidade (E) do modelo teórico. Para este modelo, encontrou-se o seguinte resultado:\

\centerline{$t = 1.329 mm$}\

\centerline{$t_w = 3.447 mm$}\

\begin{figure}[ht]
 \caption{\label{fig_ModelMetallic}Resultado do reforçador em material metálico.}
 \centering
 \includegraphics[scale=1.0]{figura/Model_Metallic1}
\end{figure}
\

\subsection{Comparação dos resultados}
A \autoref{tbl:result1_metalico} seguinte compara os resultados das espessuras do revestimento e do reforçador obtidas na análise do modelo teórico e do modelo otimizado.
\begin{table}[h]
\centering
\begin{tabular}{ccc}
\toprule
Modelo & Revestimento (mm) & Reforçador (mm) \\\midrule
Teórico & 1.524 & 3.454\\
Otimizado & 1.329 & 3.447\\
\bottomrule
\end{tabular}
\caption{Resultados de espessuras.}
\label{tbl:result1_metalico}
\end{table}

Tendo-se como base a geometria do reforçador e a densidade mássica do material considerado, calculou-se as massas das estruturas. E os resultados são conforme mostrado na \autoref{tbl:result2_metalico}.

\begin{table}[h]
\centering
\begin{tabular}{cc}
\toprule
Modelo & Massa (kg) \\ \midrule
Teórico & 0.353\\
Otimizado & 0.327\\
\bottomrule
\end{tabular}
\caption{Resultados de massa da estrutura.}
\label{tbl:result2_metalico}
\end{table}

Observa-se que foram encontradas espessuras, e consequentemente, massas das estruturas diferentes, entre os modelos teóricos e o modelo da otimização.
Em relação as espessuras, as diferenças entre os modelos foram de 12.79\% para o revestimento e 0.20\% para o reforçador, e em relação as massas de 7.37\%. Essas diferenças se devem ao fato de o modelo teórico ter considerado um reforçador ótimo com o fator de eficiência de Farrar de 0.81, e caso esse fator pudesse ser melhorado, o modelo teórico ficaria mais aproximado do modelo otimizado.

\section{Análise teórica vs. Otimização do painel reforçado}
Esta seção está dividida em três subseções:
\begin{itemize}
\item Resultados da otimização do painel reforçado fabricado em material metálico;
\item Resultados da otimização do painel reforçado fabricado em material composto;
\item Comparação dos dois itens ateriores .
\end{itemize}

\subsection{Otimização do painel reforçado em material metálico}
Desenvolveu-se um modelo de otimização do reforçador em material metálico, no qual, o material possuia as propriedades conforme mostrado na \autoref{tbl:prop_metalico}. Para este modelo, encontrou-se o seguinte resultado:\

\centerline{$t = 3.092 mm$}\

\centerline{$t_w = 2.165 mm$}\

\centerline{$t_{wBase} = 4.174 mm$}\

\begin{figure}[ht]
 \caption{\label{fig_Result1Metallic}Resultado do reforçador em material metálico.}
 \centering
 \includegraphics[scale=0.85]{figura/Results_Metallic}
\end{figure}
\

Em relação ao resultado da análise de flambagem, a \autoref{fig_Result2Metallic} mostra a resposta da estrutura com o primeiro autovalor encontrado para este reforçador em material metálico.

\begin{figure}[ht]
 \caption{\label{fig_Result2Metallic}Análise de flambagem do reforçador em material metálico ($\lambda_1 = 1.021$).}
 \centering
 \includegraphics[scale=0.85]{figura/Results2_Metallic}
\end{figure}
\

\subsection{Otimização do painel reforçado em material composto}

Desenvolveu-se um modelo de otimização do reforçador em material composto, no qual, os propriedades do material variavam com base na otimização dos parâmetros de laminação. Para este modelo, encontrou-se o seguinte resultado para as espessuras:\

\centerline{$t = 2.928 mm$}\

\centerline{$t_w = 2.611 mm$}\

\centerline{$t_{wBase} = 4.234 mm$}\

E encontrou-se o seguinte resultado para os parâmetros de laminação:\

\begin{table}[h]
\centering
\begin{tabular}{ccccc}
\toprule
Estrutura & $\xi^A_{1}$ & $\xi^A_{2}$ & $\xi^D_{1}$ & $\xi^D_{2}$ \\ \midrule
Reforçador & 0.3182 & 0.0348 & 0.0537 & -0.0625\\
Revestimento & 0.2560 & 0.0529 & -0.0851 & -0.0913\\
Base reforçador & 0.2853 & 0.0443 & 0.0215 & -0.199\\
\bottomrule
\end{tabular}
\caption{Resultados dos parâmetros de laminação.}
\label{tbl:result_qsis}
\end{table}


\begin{figure}[ht]
 \caption{\label{fig_Result1Composite}Resultado do reforçador em material composto.}
 \centering
 \includegraphics[scale=0.85]{figura/Results_Composite}
\end{figure}
\

Em relação ao resultado da análise de flambagem, a \autoref{fig_Result2Composite} mostra a resposta da estrutura com o primeiro autovalor encontrado para este reforçador em material composto.

\begin{figure}[ht]
 \caption{\label{fig_Result2Composite}Análise de flambagem do reforçador em material composto ($\lambda_1 = 1.021$).}
 \centering
 \includegraphics[scale=0.85]{figura/Results2_Composite}
\end{figure}
\

\subsection{Comparação dos resultados}

A \autoref{tbl:result1_composto} compara os resultados das espessuras do revestimento, do reforçador e da base do reforçador obtidas na otimização do modelo em material metálico e do modelo em material composto.
\begin{table}[h]
\centering
\begin{tabular}{cccc}
\toprule
Modelo & Revestimento (mm) & Reforçador (mm) & Base do reforçador (mm) \\\midrule
Metálico & 3.092 & 2.165 & 4.174 \\
Composto & 2.928 & 2.611 & 4.234\\
\bottomrule
\end{tabular}
\caption{Resultados de espessuras.}
\label{tbl:result1_composto}
\end{table}

Tendo-se como base a geometria do reforçador e a densidade mássica dos materiais, sendo $\rho_{metal} =$ $2.8$x$10^{-3} g/mm^3$ e $\rho_{composto} =$ $1.56$x$10^{-3} g/mm^3$ calculou-se as massas das estruturas. E os resultados são conforme mostrado na \autoref{tbl:result2_composto}.

\begin{table}[h]
\centering
\begin{tabular}{cc}
\toprule
Modelo & Massa (kg) \\ \midrule
Teórico & 0.551\\
Otimizado & 0.309\\
\bottomrule
\end{tabular}
\caption{Resultados de massa da estrutura.}
\label{tbl:result2_composto}
\end{table}

Observa-se que foram encontradas espessuras e massas das estruturas diferentes, entre os modelos de otimização do reforçador em material metálico e o reforçador em material composto.
Em relação a massa da estrutura, que era a função objetivo da otimização, a otimização do reforçador em material composto gerou um resultado de uma estrutura com a massa 43\% menor do que o reforçador em material composto. Isto mostra, portanto, que para a situação e modelo considerados, o reforçador em material composto é mais eficiente do que o reforçador em material metálico, visto que ambos suportam a mesma carga de flambagem, no entanto, o reforçador em material composto possui uma massa estrutural menor.


% ----------------------------------------------------------
% Finaliza a parte no bookmark do PDF
% para que se inicie o bookmark na raiz
% e adiciona espaço de parte no Sumário
% ----------------------------------------------------------
\phantompart
%
% % ---
% % Conclusão
% % ---
% % \chapter{Conclusão}
%
% - a fazer -

% % ---

% ----------------------------------------------------------
% ELEMENTOS PÓS-TEXTUAIS
% ----------------------------------------------------------
\postextual
% ----------------------------------------------------------

% ----------------------------------------------------------
% Referências bibliográficas
% ----------------------------------------------------------
\bibliography{bibliografia}

% ----------------------------------------------------------
% Glossário
% ----------------------------------------------------------
%
% Consulte o manual da classe abntex2 para orientações sobre o glossário.
%
%\glossary
%
% % ----------------------------------------------------------
% % Apêndices
% % ----------------------------------------------------------
% % % ---
% % Inicia os apêndices
% % ---
% \begin{apendicesenv}
%
% % Imprime uma página indicando o início dos apêndices
% \partapendices
%
% \chapter{apendice1}
%
% \chapter{apendice2}
%
% \end{apendicesenv}

%
%
% % ----------------------------------------------------------
% % Anexos
% % ----------------------------------------------------------
% % % ---
% % Inicia os anexos
% % ---
% \begin{anexosenv}
%
% % Imprime uma página indicando o início dos anexos
% \partanexos
%
% \chapter{anexo1}
%
% \chapter{anexo2}
%
% \end{anexosenv}

%
% %---------------------------------------------------------------------
% % INDICE REMISSIVO
% %---------------------------------------------------------------------
% \phantompart
% \printindex
% %---------------------------------------------------------------------

\end{document}
